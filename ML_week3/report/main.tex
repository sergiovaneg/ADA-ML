\documentclass{scrartcl}
\usepackage[UKenglish]{babel}
\usepackage{graphicx}

\usepackage{caption}
\usepackage{subcaption}

\usepackage{microtype}
\usepackage[inkscapeformat=eps, inkscapepath=svgdir]{svg}

\usepackage{hyperref}

\usepackage{cleveref}

\title{Advanced Data Analysis and Machine Learning - Practical Activities}
\subtitle{Sequential Data}
\author{Sergio Mauricio Vanegas Arias}
\date{\today}

\begin{document}

\maketitle

\section{Introduction}

  This week, we were asked to experiment with Recursive Neural Network architectures. Particularly, we were told to choose a type of dataset (``Time-Series forecasting'' in the case of this report) and compare the performance of the ML architectures with that of an Auto-Regressive (AR) model.

  The specific tasks are the following:

  \begin{enumerate}
    \item Import, pre-process, and visualize the dataset.
    \item Fit and compare:
    \begin{itemize}
      \item A baseline autoregressive model;
      \item A traditional RNN;
      \item An LSTM.
    \end{itemize}
    \item Interpret and comment on the results of each modelling strategy.
  \end{enumerate}

\section{Dataset Pre-Processing and Visualization}

  The chosen dataset was \href{https://www.kaggle.com/datasets/chaitanyakumar12/time-series-forecasting-of-solar-energy/}{\emph{Time series forecasting of solar energy}}, where the target variable is \emph{solar\_mw}. Before going any further, it is worth mentioning that the dataset contains a typo in row 2184 of the \emph{wind-direction} variable, where the authors inserted the text-string ``am'' instead of a numerical value. This instance was removed by hand and later interpolated.

  Out of the box, without dropping bad entries or performing any kind of interpolation, the dataset was as in \Cref{fig:timeseries_raw}. It is easy to notice that only the last fraction of the dataset contains any data beyond the target variable, so we drop all observations containing any missing value.

  \begin{figure}
    \centering
    \includesvg[width=0.8\textwidth]{figures/timeseries_raw.svg}
    \caption{Raw dataset}
    \label{fig:timeseries_raw}
  \end{figure}

  The result of this operation and interpolating linearly for any missing value within the remainder of the dataset is shown in \Cref{fig:timeseries_clean}. In this state, it is easy to visualize the periodicity and non-stationary character of the target variable, which merits the fitting of an ARIMAX model.

  \begin{figure}
    \centering
    \includesvg[width=0.8\textwidth]{figures/timeseries_clean.svg}
    \caption{Clean dataset}
    \label{fig:timeseries_clean}
  \end{figure}

\end{document}